\documentclass[12pt,a4paper]{article}
\usepackage[utf8]{inputenc}
\usepackage[T1]{fontenc}
\usepackage{amsmath}
\usepackage{amssymb}
\usepackage{graphicx}
\usepackage{bbm} % Used for indicator function
\usepackage{enumitem} % Used for enumerate w/ \alpha or roman
\usepackage{hyperref} % Used for links, such as table of contents links

\title{Probability and Measure Solutions}
\author{WispyAbyss}

% Math operators

% Definitions
\newcommand{\inner}[2]{\left\langle #1 , #2 \right\rangle}
\newcommand{\norm}[1]{\left\|#1\right\|}
\newcommand{\1}[1]{\mathbbm{1}\left\{ #1 \right\}}
\newcommand{\R}{\mathbb{R}}
\newcommand{\N}{\mathbb{N}}
\newcommand{\E}{\mathbb{E}}
\newcommand{\F}{\mathcal{F}}
\newcommand{\Z}{\mathbb{Z}}
\newcommand{\lcal}{\mathcal{L}}
\newcommand{\ucal}{\mathcal{U}}
\newcommand{\tcal}{\mathcal{T}}
\newcommand{\Prob}{\mathbb{P}}
\newcommand{\VCdim}{\text{VCdim}}
\newcommand{\st}{\text{s.t.}}
\newcommand{\roundUp}[1]{\left\lceil #1 \right\rceil}
\newcommand{\roundDown}[1]{\left\lfloor #1 \right\rfloor}
\newcommand{\sign}{\text{sign}}
\newcommand{\ceil}[1]{\left\lceil#1\right\rceil}
\renewcommand{\div}{\text{div}}
\newcommand{\curl}{\text{curl}}
\newcommand{\trace}{\text{trace}}
\newcommand{\grad}{\text{grad}}


\setcounter{secnumdepth}{0}

\begin{document}
\maketitle
	
\tableofcontents
	
\section{Forward}
This document will contain notes and solutions corresponding to Probability and Measure, Third Edition, by Patrick Billingsley  [\href{https://www.amazon.com/PROBABILITY-MEASURE-WILEY-MATHEMATICAL-STATISTICS/dp/8126517719/ref=sr_1_2?crid=3IVF52UANVNQC&keywords=Probability+and+Measure+by+Patrick+Billingsley&qid=1694149664&s=books&sprefix=probability+and+measure+by+patrick+billingsley%2Cstripbooks%2C143&sr=1-2}{amazon}].

\section{Chapter 1.1 - Borel's Normal Number Theorem}
\subsection{Notes}
For a complete understanding of probability, you need to understand an infinite number of events as well as a finite number of events. We try and present why that must be so here.

\subsubsection{The Unit Interval} 
We take the length of an interval $I = (a,b] = b - a$. Note, for $A$ a disjoint set of intervals in $(0,1]$, we have that $P(A)$ is well defined. If $B$ is a similar disjoint set, and is disjoint from $A$, $P(A + B) = P(A) + P(B)$ is well defined as well. Note - we haven't defined anything for intersections yet. These definitions can also directly stem from the Riemann integral of step functions.
\\\\
The unit interval can give the probability that a single particle is emitted in a unit interval of time. Or a single phone call comes in. However, it can also model an infinite coin toss. This is done as follows - for $\omega \in (0,1]$, define:
$$
	\omega = \sum_{n=1}^\infty \frac{d_n(\omega)}{2^n}
$$
Where $d_n(\omega)$ is $0$ or $1$, and comes from the binary expansion of $\omega$. We take $\omega$ as the non terminating representation. Note, we were particular when we defined intervals as half inclusive. Examine the set of $\omega$ for which $d_i(\omega) = u_i$ for $i = 1, \cdots, n$, $u_i \in \{0,1\}$. We have that:
$$
	\sum_{i=1}^n \frac{u_i}{2^i} < \omega \leq \sum_{i=1}^n \frac{u_i}{2^i} + \sum_{i=n+1}^\infty \frac{1}{2^i}
$$
We cannot have the lower extreme value, as this would imply $\omega$ takes on its terminating binomial representation, which is what we said we would not do. This is our first taste, I guess, of measure $0$ sets, we we still have:
$$
	\Prob\left[\omega: d_i(\omega) = u_i, i = 1, \cdots, n \right] = \frac{1}{2^n}
$$
Note, probabilities of various familiar events can be written down immediately. Ultimately, note, however, each probability is the sum of disjoint dyadic intervals of various ranks $k$. Ie, all the events are still well defined by our probability definition above. We have:
$$
	\Prob\left[\omega : \sum_{i=1}^n d_i(\omega) = k\right] = {n \choose k} \frac{1}{2^n}
$$


\subsection{Solutions}
	
\end{document}



