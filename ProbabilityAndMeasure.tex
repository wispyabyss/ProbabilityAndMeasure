\documentclass[12pt,a4paper]{article}
\usepackage[utf8]{inputenc}
\usepackage[T1]{fontenc}
\usepackage{amsmath}
\usepackage{amssymb}
\usepackage{graphicx}
\usepackage{bbm} % Used for indicator function
\usepackage{enumitem} % Used for enumerate w/ \alpha or roman
\usepackage{hyperref} % Used for links, such as table of contents links

\title{Probability and Measure Solutions}
\author{WispyAbyss}

% Math operators

% Definitions
\newcommand{\inner}[2]{\left\langle #1 , #2 \right\rangle}
\newcommand{\norm}[1]{\left\|#1\right\|}
\newcommand{\1}[1]{\mathbbm{1}\left\{ #1 \right\}}
\newcommand{\R}{\mathbb{R}}
\newcommand{\N}{\mathbb{N}}
\newcommand{\E}{\mathbb{E}}
\newcommand{\F}{\mathcal{F}}
\newcommand{\Z}{\mathbb{Z}}
\newcommand{\lcal}{\mathcal{L}}
\newcommand{\ucal}{\mathcal{U}}
\newcommand{\tcal}{\mathcal{T}}
\newcommand{\Prob}{\mathbb{P}}
\newcommand{\VCdim}{\text{VCdim}}
\newcommand{\st}{\text{s.t.}}
\newcommand{\roundUp}[1]{\left\lceil #1 \right\rceil}
\newcommand{\roundDown}[1]{\left\lfloor #1 \right\rfloor}
\newcommand{\sign}{\text{sign}}
\newcommand{\ceil}[1]{\left\lceil#1\right\rceil}
\renewcommand{\div}{\text{div}}
\newcommand{\curl}{\text{curl}}
\newcommand{\trace}{\text{trace}}
\newcommand{\grad}{\text{grad}}


\setcounter{secnumdepth}{0}

\begin{document}
\maketitle
	
\tableofcontents
	
\section{Forward}
This document will contain notes and solutions corresponding to Probability and Measure, Third Edition, by Patrick Billingsley  [\href{https://www.amazon.com/PROBABILITY-MEASURE-WILEY-MATHEMATICAL-STATISTICS/dp/8126517719/ref=sr_1_2?crid=3IVF52UANVNQC&keywords=Probability+and+Measure+by+Patrick+Billingsley&qid=1694149664&s=books&sprefix=probability+and+measure+by+patrick+billingsley%2Cstripbooks%2C143&sr=1-2}{amazon}].

\section{Chapter 1.1 - Borel's Normal Number Theorem}
\subsection{Notes}
For a complete understanding of probability, you need to understand an infinite number of events as well as a finite number of events. We try and present why that must be so here.

\subsubsection{The Unit Interval} 
We take the length of an interval $I = (a,b] = b - a$. Note, for $A$ a disjoint set of intervals in $(0,1]$, we have that $P(A)$ is well defined. If $B$ is a similar disjoint set, and is disjoint from $A$, $P(A + B) = P(A) + P(B)$ is well defined as well. Note - we haven't defined anything for intersections yet. These definitions can also directly stem from the Riemann integral of step functions.
\\\\
The unit interval can give the probability that a single particle is emitted in a unit interval of time. Or a single phone call comes in. However, it can also model an infinite coin toss. This is done as follows - for $\omega \in (0,1]$, define:
$$
	\omega = \sum_{n=1}^\infty \frac{d_n(\omega)}{2^n}
$$
Where $d_n(\omega)$ is $0$ or $1$, and comes from the binary expansion of $\omega$. We take $\omega$ as the non terminating representation. Note, we were particular when we defined intervals as half inclusive. Examine the set of $\omega$ for which $d_i(\omega) = u_i$ for $i = 1, \cdots, n$, $u_i \in \{0,1\}$. We have that:
$$
	\sum_{i=1}^n \frac{u_i}{2^i} < \omega \leq \sum_{i=1}^n \frac{u_i}{2^i} + \sum_{i=n+1}^\infty \frac{1}{2^i}
$$
We cannot have the lower extreme value, as this would imply $\omega$ takes on its terminating binomial representation, which is what we said we would not do. This is our first taste, I guess, of measure $0$ sets, we we still have:
$$
	\Prob\left[\omega: d_i(\omega) = u_i, i = 1, \cdots, n \right] = \frac{1}{2^n}
$$
Note, probabilities of various familiar events can be written down immediately. Ultimately, note, however, each probability is the sum of disjoint dyadic intervals of various ranks $k$. Ie, all the events are still well defined by our probability definition above. We have:
$$
	\Prob\left[\omega : \sum_{i=1}^n d_i(\omega) = k\right] = {n \choose k} \frac{1}{2^n}
$$
All these results have been for finitely many components of $d_i(\omega)$. What we are interested in, however, is properties of the entire sequence of $\omega = (d_1(\omega), d_2(\omega), \cdots)$. 

\subsubsection{The Weak Law of Large Numbers}
What I like about this chapter, is to me - it \textit{emphasizes} the connection between the \textit{structure of real numbers}, and probability. At the end of the day - probability can be seen as just extracting properties of \textit{frequency} over the real numbers, to be understood as probabilistic statements. However, with just our basic real numbers - we can't really prove a lot of properties about infinite things. That is when measure theory comes in later. However, for now, we look at what we can prove - and that starts with the weak law of large numbers. We have:

\paragraph{Theorem 1.1 - The Weak Law of Large Numbers} For each $\epsilon$:
$$
	\lim_{n \to \infty} \Prob\left[
	\omega: \left|\frac{1}{n}\sum_{i=1}^n d_i(\omega) - \frac{1}{2}\right| \geq \epsilon
	\right] = 0
$$
Probabilistically - this is saying that if $n$ is large, then there is a small probability that the fraction/relative \textit{frequency} of heads in $n$ tosses will deviate much from $1/2$. Think about it as a statement over the real numbers as well - it is also interesting. Ultimately, the intervals containing $\omega$ that do not satisfy the above are getting smaller and smaller and smaller. We formalize this with the following concept:
\\\\
As $d_i(\omega)$ are constant over each dyadic interval of rank $n$ if $i \leq n$, the sums $\sum_{i=1}^n d_i(\omega)$ are also constant over rank $n$. Thus, the set in the theorem is just a disjoint union of dyadic intervals of rank $n$. Note - the theorem is saying, that the total weight given to those intervals gets smaller and smaller as $n$ goes to infinity.
\\\\
Now, we go over how to prove the theorem. It relies on rademacher variables:
$$
	r_n(\omega) = 2d_n(\omega) - 1
$$
These are $\pm 1$ when $d_n = 1/0$. Note, these have the same "being constant on dyadic intervals" properties as $d_n(\omega)$. We define:
$$
	s_n(\omega) = \sum_{i=1}^n r_i(\omega)
$$
And so, our theorem is equivalent to proving:
$$
	\lim_{n \to \infty} \Prob\left[
	\omega: \left|\frac{1}{n} s_n(\omega)\right| \geq \epsilon
	\right] = 0
$$
Note, rademacher functions also have interpretations, probabilistically, of random walks and such. With these variables, we can ultimately find properties, going all the way to:
$$
	\int_0^1 s_n^2(\omega) = n
$$
However, what interests me is the following: Chebyshev's Lemma, but as a property of the real numbers. We have:

\paragraph{Lemma - Chebyshev's Inequality} If $f$ is a nonnegative step function, then $[\omega: f(\omega) \geq \alpha]$ is for $\alpha > 0$ a finite union of intervals, and:
$$
	\Prob\left[\omega: f(\omega) \geq \alpha\right] \leq \frac{1}{\alpha} \int_0^1 f(\omega) d\omega
$$
Proof: Note, it is all just properties of step functions. Let $c_j$ correspond to the step intervals $(x_{j-1},x_j]$, and let $\sum'$ be the sum over $c_j \geq \alpha$. Then, we have quite easily:
$$
	\int_0^1 f(\omega) d\omega =
	\sum c_j (x_j - x_{j-1}) \geq 
	\sum' c_j (x_j - x_{j-1}) \geq 
	\sum' \alpha (x_j - x_{j-1}) = \alpha \Prob\left[\omega: f(\omega) \geq \alpha\right]
$$ 
Thus, we have Chebyshev's inequality, and with it, we can easily prove the Weak Law of Large Numbers. However - it is important to note - these are \textit{properties over the real numbers}, as much as they are probabilistic properties.

\subsubsection{The Strong Law of Large Numbers}
Just to first formalize some terms - the frequency of $1$ in $\omega$ is $\sum_{i=1}^n d_i(\omega)$, the relative frequency is that number normalized, ie $\frac{1}{n} \sum_{i=1}^n d_i(\omega)$, and the asymptotic relative frequency is the limit. We can derive, with some technical tools outside of discrete probability theory, results on the set:
$$
	N = \left[\omega : \lim_{n \to \infty} \frac{1}{n} \sum_{i=1}^n d_i(\omega) = 1/2\right]
$$
We call this the set of normal numbers $N$. The tools themselves are the concepts of negligibility. A set $A$ is negligible if for every $\epsilon > 0$, there is a countable number of intervals (not necessarily disjoint) such that:
$$
	A \subset \bigcup_k I_k \quad\quad\quad \sum_k I_k = \sum_k b_k - a_k < \epsilon
$$
For one - I like to note here interpretations. Essentially - if $A$ is negligible, it is a practical impossibility that $\omega$ randomly drawn will lie within $A$. And if $A^c$ is negligible, it is a practical certainty that $\omega$ randomly drawn will lie within $A$. These are just how they should be understood - and these understandings are reasonable, as the total "length" that $A$ takes up can be understood to be incredibly incredibly small.
\\\\
Some properties of negligibility - note, these are the standard properties, stemming from infinite sums $(1/2^k)$ summing to values less than $\epsilon$. Individual points are negligible, and so to thus are countable sets. So to are countable unions of countable sets.
\\\\
With these properties - we understand that the property of our model not including $\omega$ with a terminating sequence (all $0$ ending) is not a short coming. These $\omega$ form a countable set - and so, they can be considered negligible.

\paragraph{Theorem 1.2} The set of normal numbers $N$ has negligible complement.
\\\\
\textbf{Proof} As an aside - we note that this proof is stronger than just the negligibility properties we noted above. This is because $N^c$ is not countable. The set of $d_i(\omega) = 1$ unless $i$ is a multiple of $3$ clearly belongs to $N$ - as for each $n$, $n^{-1}\sum_{i=1}^n d_i(\omega) \geq 2/3$. However, note this set is uncountable (diagonalization argument).
\\\\
Note, the proof relies on equivalently defining $N$ as:
$$
	N = \left[\omega : \lim_{n \to \infty} \frac{1}{n} s_n(\omega) = 0\right]
$$
Then, we can again make use of Chebyshev's Inequality (step function version) to find that:
$$
	\Prob\left[\omega : |s_n(\omega)| \geq n\epsilon\right] \leq 
	\frac{1}{n^4\epsilon^4}\int_0^1 s_n^4(\omega) d\omega =
	\frac{n + 3n(n-1)}{n^4\epsilon^4} \leq
	\frac{3}{n^2\epsilon^4}
$$
Where the last step is just via an in depth (but simple) investigation of the integrals of multiplications of rademacher variables. With this property, we can find that if $A_n = [\omega: |n^{-1}s_n(\omega)| \geq \epsilon_n]$, then we have a sequence of $\epsilon_n$ such that $P(A_n) \leq 3\epsilon_n^{-4}n^{-2}$, and we can find such a sequence such that:
$$
	\sum_n \Prob\left[A_n\right] < \infty
$$
The final step to proving the theorem is noting that:
$$
	\bigcap_{n = m}^\infty A_n^c \subset N \implies N^c \subset \bigcup_{n=m}^\infty A_n
$$
Which will ultimately prove the theorem. Note - a lot of details are left out, but I do not consider them important. You should be able to fill in. These are just the major strokes, outlining the proof. It essentially hinges on our integral value, and the relationship between $A_n$ and the set of normal number $N$. qed.
\\\\
So, we have $N^c$ is negligible. But, can we have that $N$ itself is negligible? Well, we could say no - using our "practically impossible" notions, and noting that for $\omega \in [0,1]$ randomly drawn, it must be in $[0,1]$, and $N^c \cup N = [0,1]$. But, that is not rigorous. And so, the following theorem will give us our initial basis of \textit{measure}, and also help us note that $N$ is not negligible.

\paragraph{Theorem 1.3 - Lebesgue Measure Starting Point}
\begin{enumerate}
	\item If $\bigcup_k I_k \subset I$, and the $I_k$ are disjoint, then $\sum_k |I_k| \leq |I|$
	\item If $I \subset \bigcup I_k$ (the $I_k$ need not be disjoint), then $|I| \leq \sum_k |I_k|$
	\item If $\bigcup I_k = I$, and the $I_k$ are disjoint, then $|I| = \sum_k |I_k|$
\end{enumerate}
Note, this Theorem is true for countably infinite intervals as well. \textbf{Proof:} Note that the third part follows directly from $(1)$ and $(2)$. We start with the finite cases. For $(1)$, we can prove by induction on the number of intervals $n$. It is clearly true for $n = 1$, and it is a fairly simple induction hypothesis to prove in general. We similarly have the same for $(2)$.
\\\\
The difficult part comes when going to infinite intervals. For $(1)$, it is a simple limit, ie:
$$
	\sum_k |I_k| = \lim_{n \to \infty} \sum_{k=1}^n |I_k|
$$
Note, each sum is less than $|I|$, as the finite case to $1$ applies for each finite sum. And so, the inequality can be expanded to the limit. However - we can't do that for $(2)$. Ultimately, the difference between the two cases is the inclusion of unions. We note:
$$
	\bigcup_k I_k \subset I \implies \bigcup_{k=1}^n I_k \subset I
$$
Ie, the inclusion is true for every subset. However, we \textit{do not} necessarily have:
$$
	I \subset \bigcup_k I_k \implies I \subset \bigcup_{k=1}^n I_k
$$
Note that in the following way: $I = (a,b]$. We have that $I_i = (a + 1/i, b]$. We do indeed have that:
$$
	I \subset \bigcup_k I_k
$$
As if you take $x \in I$, $a < x \leq b$, and so we must have for $i$ large enough, $a + 1/i < x \leq b$, and so $x \in I_i$. However, note that the inclusion is not actually true for a specific finite subunion. So, we need to take a different strategy to prove the infinite case. This comes from dealing with \textit{open covers of compact spaces}, and relying on the Heine-Borel theorem, which says that intervals $[a,b]$ are indeed compact. In this case, we are able to bridge between infinite unions and finite unions - as we can take a finite sub cover of an open cover on compact spaces. We prove the theorem essentially for $[a + \epsilon,b]$, that:
$$
	|I| - \epsilon = b - (a + \epsilon) \leq \sum_k |I_k| + \epsilon
$$
However, as the $\epsilon$ is arbitrary, we can conclude the fact for the infinite case as well. qed.
\\\\
Note - this implies that $N$ is not negligible. As it it was, $[0,1]$ would be negligible, but that is incorrect by the above, as any open covering must have total sum at least $1$, ie, the total sum is not smaller than arbitrary $\epsilon$.

\subsubsection{The Measure Theory of Diophantine Approximation}
This section is just additional, so my notes here are sparse. However, I do read through it, and record the theorems, plus some notes I have on them.

\paragraph{Theorem 1.4} If $x$ is irrational, there are infinitely many irreducible fractions $p/q$ such that:
$$
	\left|x - \frac{p}{q}\right| < \frac{1}{q^2}
$$
Honestly, this proof is so good. I like it a lot - it is pretty clever. However, I don't just want to copy it down here - it is in the book. I'm not sure if there is any broad message I can glean from it - just that, it is a property of the real numbers. It just hinges on the following fact (which itself is pretty difficult to prove), that for every $Q$ positive integer, there is an integer $q < Q$ and corresponding $p$ such that:
$$
	\left|x - \frac{p}{q} \right| < \frac{1}{q Q} \leq \frac{1}{q^2}
$$
Note, this is true for $x$ rational or irrational. However, we have an infinite number of such irreducible fractions for the irrational case, and the contradiction derived in the book is nice as well. Anyway - read the book for this. qed.
\\\\
Anyway, the above essentially means that, apart from a negligible set of $x$, each real number has an infinite set of irreducible rationals such that the bounds in Theorem 1.4 are true. We now consider a generalization - when can we tighten the inequality in Theorem 1.4 - Consider:
$$
	\left|x - \frac{p}{q}\right| < \frac{1}{q^2\varphi(q)}
$$
Let $A_\varphi$ consist of the real $x$ for which the above has infinitely many irreducible solutions. Under what conditions on $\varphi$ will $A_\varphi$ have negligible complement? Note that if $\varphi(q) < 1$, then the condition is weaker than Theorem 1.4, and so $A_\varphi$ has negligible complement immediately. It becomes interesting if $\varphi(q) > 1$. We will later prove the theorem:

\paragraph{Theorem 1.5} Suppose that $\varphi$ is positive and nondecreasing. If:
$$
	\sum_q \frac{1}{q\varphi(q)} = \infty
$$
Then $A_\varphi$ has negligible complement. We will prove this later, but we can now prove:

\paragraph{Theorem 1.6} Suppose that $\varphi$ is positive. If
$$
	\sum_q \frac{1}{q\varphi(q)} < \infty
$$
Then $A_\varphi$ is negligible.
\\\\
We will go over the proof soon for this theorem. However - just note what the theorems are saying. Note that in the second - $\varphi(q)$ must be growing quite quickly. We need the denominator to be quite large, so that the infinite sum is ultimately finite. However, in theorem 1.5, we don't want the $\varphi(q)$ to be too large, lest the sum actually does become finite. Ultimately - both theorems are conditions on how $\varphi(q)$ grows. Which, ultimately does make sense. If $\varphi(q)$ grows to large - it becomes unreasonable to expect our condition to hold infinitely many times. If I ever encounter such situations, where I might want to examine the growth of a function $\varphi$ - I think examining whether the infinite sum of $1/\varphi$ equals infinity or not is often a good property that is related to the growth of a function.
\\\\
\textbf{Proof of Theorem 1.6} I'll give the full proof here, as it is interesting to me, and rather short. We want to show that $A_\varphi$ is negligible. Well, given that the sum is finite, there is a $q_0$ large enough such that the tail sum $\sum_{q \geq q_0} \frac{1}{q\varphi(q)} < \epsilon/4$. If $x \in A_\varphi$, then our definition holds for some $q \geq q_0$, and as $0 < x < 1$, we have that the corresponding $p$ lies in $0 \leq p \leq q$. Thus, we have that:
$$
	A_\varphi \subset \bigcup_{q \geq q_0} \bigcup_{p = 0}^q 
	\bigg(\frac{p}{q} - \frac{1}{q^2\varphi(q)}, \frac{p}{q} + \frac{1}{q^2\varphi(q)}\bigg]
$$
Which stems from every $x \in A_\varphi$ being in the right expression, given that $x$ is within one of the intervals on the right by the property we just described. Now, we have a covering interval - we just need to find the length of it. Note, by assumption, we have all of our $q$ must satisfy $q \geq 1$ (or, we can add that in). And so, we have the sum of the intervals is:
$$
	\sum_{q \geq q_0} \sum_{p = 0}^q \frac{2}{q^2\varphi(q)} =
	\sum_{q \geq q_0} \frac{2(q + 1)}{q^2\varphi(q)} \leq
	\sum_{q \geq q_0} \frac{2(q + q)}{q^2\varphi(q)} \leq
	\sum_{q \geq q_0} \frac{4}{q\varphi(q)} < \epsilon
$$
And thus, $A_\varphi$ is negligible. qed.

\subsection{Problems}
\subsubsection{1.1 Infinite Independent Events on a Discrete Space (are impossible)} 
\begin{enumerate}
	\item As for why the existence of an infinite sequence of independent events each with probability $1/2$ in a discrete probability space would make the section superfluous - I think this is because, in the section, we \textit{rely} on the uncountability of the real numbers. This allows us to make notions like negligible, which helps us make Borel's Number Theorem (the Strong Law of Large numbers). We could then just handle infinite cases with a countable, discrete space - which would make the section unnecessarily in depth ("superfluous").
	\\\\
	As for why a discrete space cannot have an infinite sequence of independent events. Note, we can partition the space into sets $A_1 \cap A_2$, $A_1 \cap A_2^c$, $A_1^c \cap A_2$, and $A_1^c \cap A_2^c$. Note that each has probability $2^{-2}$, by independence. And so, each countable point $\omega$ belongs to one of these sets, and $P(\omega) \leq 2^{-2}$. We can continue on for arbitrary $2^k$ partitions, each of probability at most $2^{-k}$. Thus, we find that $P(\omega) = 0$ for each point. This is a contradiction, as:
	$$
		\sum P(\omega) = 1 \neq 0 = \sum_\omega 0 = \sum P(\omega)
	$$
	
	\item This portion draws the same contradiction as above, namely, $P(\omega) = 0$ for all $\omega$. Each $\omega$ belongs to a sequence of $A_1, A_2, A_3^c$, something like that. Let $t_i = p_i/1-p_i$ that corresponds to $\omega \in A_i$ or $\omega \in A_i^c$. We find:
	$$
		P(\omega) \leq \prod_{i=1}^n t_i \leq \exp\left(-\sum_{i=1}^n (1-t_i)\right)
	$$
	Where the second step notes a property for $t_i \in [0,1]$. Note, it is clear that the above is bounded by:
	$$
		\leq \exp\left(-\sum_{i=1}^n \alpha_i\right)
	$$
	If $\sum_n \alpha_n$ diverges, the above goes to $0$, and we can conclude:
	$$
		P(\omega) = 0
	$$
	Which again, draws out our contradiction. qed.
\end{enumerate}

\subsubsection{1.2 Normal Numbers and Complements are Dense in $(0,1]$} Show that $N$ and $N^c$ are dense in $(0,1]$. Recall, the definition of \textit{dense} is that $N$ is dense in $(0,1]$ if for each $x \in (0,1]$, and each interval $J$ containing $x$, there is a $y \in N$ such that $y \in J$.
\\\\
Take $\omega \in (0,1]$. We note $\omega$ has some form:
$$
	(d_1(\omega),d_2(\omega), \cdots)
$$
Note, the problem is equivalent to saying that if $\omega \in N$, can we find $x \in N^c$ arbitrarily close to $\omega$, and vice versa, for $\omega \in N^c$ and $x \in N$. We first assume $\omega \in N$. We can easily find an $x \in N^c$, such that $x$ is arbitrarily close to $\omega$. Just take the first $k$ elements matching, so that we are within $\frac{1}{2^k}$ of $\omega$, and continue with ones. Clearly, such an $x$ can be arbitrarily close to $\omega$, and within $N^c$. So, $N^c$ is clearly dense.
\\\\
For the other direction, take $\omega \in N^c$, and now, again, match $x$ on the first $k$ elements of the binary expansion. For the remainder, oscillate between $1$ and $0$. Clearly, $x \in N$, and arbitrarily close to $\omega$. qed.

\subsubsection{1.3 Trifling Set Properties} \textbf{Definition:} Define a set $A$ to be \textit{trifling} if for each $\epsilon$ there exists a \textit{finite} sequence of intervals $I_k$ satisfying that they cover $A$ and interval sum less than $\epsilon$. Recall, from Calculus on Manifolds - this is essentially content $0$. 
\begin{enumerate}
	\item A trifling set is also negligible. This must is clear - take the remaining infinite intervals as ones that sum up to less than a small enough $\epsilon'$.
	
	\item Show that the \textit{closure} of a trifling set is also trifling. Recall, the closure is all points that are not exterior to $A$ - exterior meaning that they have open neighborhoods not intersecting $A$. Well, take a finite covering less than $\epsilon/2$ of $A$. We define:
	$$
		A \subset \bigcup_{k=1}^n I_k \quad\quad\quad \sum_{k=1}^n I_k < \frac{\epsilon}{2} \quad\quad\quad
		I_k' = \bigg(a_k - \frac{\epsilon}{2^{k+2}}, b_k + \frac{\epsilon}{2^{k+2}}\bigg]
	$$
	$$
		\implies \sum_{k=1}^n I_k' < \epsilon
	$$
	Note that $I_k'$ covers $\overline{A}$. Take $x \in \overline{A}$. By definition, we have $\left(x - \frac{\epsilon}{2^{n+2}}, x + \frac{\epsilon}{2^{n+2}}\right)$ intersects $A$, and so coincides with some $I_k$, and so must be contained within $I_k'$. Thus, it is clear that $I_k'$ covers $\overline{A}$, and so $\overline{A}$ is trifling as well.
	
	\item The rationals in $(0,1]$ are bounded and negligible (being countable), but not trifling. Assume we have a covering of the rationals that is finite and sums to less than $\epsilon$. Note, we can take the covering to be restricted to $(0,1]$, as all the rationals are in $(0,1]$. For $\epsilon < 1$ - Theorem 1.3.2 implies that these intervals do not cover all of $(0,1]$. Note, if they don't cover a rational, we are done. We now note that these sets must not cover some interval of non negligible length - if they covered every such interval, there sum would be $1$. This interval contains a rational, which contradicts the set being covering.
	
	\item Show that the closure of a negligible set may not be negligible. Again, the closure of the rationals in $(0,1]$ is $(0,1]$, which is not negligible.
	
	\item Show that finite unions of trifling sets are trifling, but that this can fail for countable unions. Fail for countable unions - take the union of each rational, which is a countable union of trifling singleton sets - by the above, this is negligible, and not trifling. Note, for finite unions of trifling sets - say $k$ such sets - take the covering of size $\frac{\epsilon}{2^k}$, and note that the union of these intervals is a finite covering of total length less than $\epsilon$.
\end{enumerate}

\subsubsection{1.4 Trifling Sets In Base $r$}
\begin{enumerate}
	\item First thing to note. We can look at $A_r(i)$ as iteratively removing intervals from $(0,1]$, where step $k$ corresponds to removing the numbers whose expansions do not contain $i$ for the first $k - 1$ digits, but contain $i$ at digit $k$. At step $k$, we remove $(r-1)^{k-1}$ intervals (corresponding to the $r-1$ possible digits in the first $k-1$ spaces) of length $\frac{1}{r^k}$ (corresponding to the length of the interval starting with $i \in [r-1]$ in the $kth$ entry going to $i + 1$ in the $kth$ entry). We find, that the total length of the disjoint intervals removes is:
	$$
		\sum_{k=1}^\infty \frac{(r-1)^{k-1}}{r^k} = \frac{1}{r} \sum_{k=1}^\infty \frac{(r-1)^{k-1}}{r^{k-1}} =
		\frac{1}{r} \sum_{k=0}^\infty \frac{(r-1)^k}{r^k} = \frac{1}{r} * \frac{1}{1/r} = 1
	$$
	Where the second to last step is the sum of a geometric series. So, at the very least, we have that it could be possible that $A_r(i)$ is trifling.
	\\\\
	Here is how it is trifling. We \textit{know} that a finite amount of points is trifling. As the above sum equals $1$ - if we go far enough, the amount removed will be arbitrarily close to $1$. Say, within $\epsilon/2$ of $1$. And so, the remaining \textit{intervals} that are uncovered must have at most a total length of $\epsilon/2$. We can cover those intervals with the intervals themselves. Frankly, I think that is enough. Note, of course, at each step we remove an interval that looks like $(a,b]$. And so, the remaining intervals should be of the form $(c,d]$ as well. Everything should be nice, as the intervals in our iteration are disjoint. I literally think that is it. qed.
	
	\item We want to find a trifling set $A$ such that every point in the unit interval can be represented in the form $x + y$ with $x$ and $y$ in $A$.
	
	\item Let $A_r(i_1, \cdots, i_k)$ consist of the numbers in the unit interval in whose base $r$ expansion the digits $i_1, \cdots, i_k$ nowhere appear consecutively in that order. Show that it is \textit{trifling}.
	\\\\
	The first observation I have made: if we have that $i_1, \cdots, i_k$ are all equal, whereas $j_1, \cdots, j_k$ are an arbitrary sequence of digits, we have that:
	$$
		|A_r(j_1, \cdots, j_k)| \leq |A_r(i_1, \cdots, i_k)|
	$$
	The reason is the following: for the first $n$ digits of the base $r$ expansion, there are more numbers without $i_1, \cdots, i_k$ appearing consecutively then there are numbers without $j_1, \cdots, j_k$ appearing consecutively. Consider the following example: Base 3, with $n = 3$. We have the following possible sequences:
	$$
	000 \quad\quad 001 \quad\quad 002 \quad\quad 010 \quad\quad 011 \quad\quad 012 \quad\quad 020 \quad 021 \quad 022
	$$
	$$
	100 \quad\quad 101 \quad\quad 102 \quad\quad 110 \quad\quad 111 \quad\quad 112 \quad\quad 120 \quad 121 \quad 122
	$$
	$$
	200 \quad\quad 201 \quad\quad 202 \quad\quad 210 \quad\quad 211 \quad\quad 212 \quad\quad 220 \quad 221 \quad 222
	$$
	Consider the count of sequences above without $11$. There are 22 such sequences. Now, count the sequences without $12$. There are $21$ such sequences. Note, above, we have that the sequence $111$ has $11$ at the start, and $11$ at the end, but only takes up one entry. However, we have that $12X$ and $Y12$ can never be the same, and so there are two such sequences taken up. We can expand this concept in general - when $i_1, \cdots, i_k$ are all equal, we get the most \textit{collisions} between sequences with digits $i_1, \cdots, i_k$ starting at the possible $n - k + 1$ starting points. And so, if we find that $A_r(i_1, \cdots, i_k)$ is trifling for $i_1 = \cdots = i_k$, we can conclude that $A_r(j_1, \cdots, j_k)$ is trifling in general.
	\\\\
	To be honest, I am going to skip this one, because I am getting nowhere. However, here is the work I've done so far, for what it is worth. I have made the following definitions:
	$$
		S_{k,n} = \left\{\text{The length $n$ sequences that contain the digit $d$ repeated $k$ times}\right\}
	$$
	$$
		A_{t,n} = \left\{\text{The length $n$ sequences where $d$ repeated $k$ times first appears at pos $t$}\right\}
	$$
	And so, with these definitions, we have:
	$$
		S_{k,n} = \bigcup_{t=1}^{n-k+1} A_{t,n} \implies
		|S_{k,n}| = \sum_{t=1}^{n-k+1} |A_{t,n}|
	$$
	Where the first step is just definitional, as if $d$ is repeated $k$ times, that subsequence first starts at position $1, 2$, or up to position $n - k + 1$. The second step comes from noting that the sets $A_{t,n}$ are disjoint - if the sequence first starts at position $t$, it \textit{does not} start at position $t' \neq t$. And so, if we can find that $\lim_{n \to \infty} \frac{|S_{k,n}|}{r^n} = 1$, then for $n$ large enough, it will equal $1 - \epsilon$, and we can take a \textit{finite} number of intervals to cover the remaining intervals of total length $\epsilon$ that are not represented in the \textit{finite} union of intervals $\bigcup_{t=1}^{n-k+1} A_{t,n}$. The difficult part is actually finding what the above is in terms of numbers. However, I now note that:
	$$
		|A_{t,n}| = r^{n - (t + k - 1)} \cdot 1^k \cdot (r - 1) \cdot (r^{t - 2} - |S_{k,t-2}|)
	$$
	This comes from examining what each of the possible digits in our expansion of $x \in A_{t,n}$ can be. The remaining $n - (t + k - 1)$ after the digits $d$ repeated $k$ times starting at position $t$ can be anything we want. This gives us our first element in the product. $1^k$ refers to the $k$ digits starting at $t$ must all be $d$. $(r-1)$ refers to position $t - 1$ - that \textit{must} be any number other than $d$. If it is $d$, then we get $x \in A_{t-1,n}$, which is incorrect. Finally, the remaining first $t-2$ entries can be any sequence at all, except for a sequence of $d$ repeated $k$ times. The count of those sequences is removed from the total $r^{t-2}$ sequences possible. And so, we find that essentially, both sides are equal. If we make any sequence described on the right side, we have that it is within $A_{t,n}$. And, any sequence in $A_{t,n}$ can be described on the right side. And so now, the work remains to just simplify the calculation of $|S_{k,n}|$. Note, we could actually calculate this value by a recursive algorithm. That might help us.
	\\\\
	Big Note: The calculation of $|A_{t,n}|$ assumes somethings, like the existence of position $t - 1$ (which is not there if $t = 1$) or $t > 2$ for $r^{t - 2}$. Just make sure to keep these exceptions in mind. Anyway. We make a hand wavy assumption - that $r^n - |S_{k,n}| > r^{k-1}$. Note, $r$ is some base $n$ digit, and so $r^{k-1}$ is just some constant. And while we want to eventually prove that the ratio is equal to $1$ in limit - ultimately, there will always be some constant distance between $r^n$ and $|S_{k,n}|$. And this constant will continue to grow to infinity. Anyway, for $n$ large enough, I think it is clear. And so, being hand wavy, and including all terms, although they might not be present, we have:
	$$
		\lim_{n \to \infty} \frac{|S_{k,n}|}{r^n} \approx 
		\lim_{n \to \infty} \frac{\sum_{t=1}^{n - k + 1} r^{n - (t + k - 1)} \cdot (r - 1) \cdot (r^{t - 2} - |S_{k,t-2}|)}{r^n}
	$$
	$$
		=
		\lim_{n \to \infty} \frac{r^{n - k + 1}(r-1)\sum_{t=1}^{n - k + 1} r^{-t} \cdot (r^{t - 2} - |S_{k,t-2}|)}{r^n}
	$$
	$$
		= r^{-k + 1} (r-1) \lim_{n \to \infty} \sum_{t=1}^{n - k + 1} r^{-t} \cdot (r^{t - 2} - |S_{k,t-2}|)
	$$
	Now, making use of our hand waviness, we have that:
	$$
		\geq r^{-k + 1} (r-1) \lim_{n \to \infty} \sum_{t=1}^{n - k + 1} r^{-t} r^{k-1} =
		(r-1) \lim_{n \to \infty} \sum_{t=1}^{n - k + 1} r^{-t}
	$$
	Now, we can make use of our geometric series, and have that the above equals:
	$$
		= (r-1) * \frac{1}{r-1} = 1
	$$
	And so yes, the limit does indeed equal $1$. Unraveling everything we said above, this allows us to conclude that $A_r(i_1, \cdots, i_k)$ is indeed trifling. Now, there is a saying, that a monkey typing at random for infinity will ultimately write Shakespeare. Well, we can let every word be a digit in some base 10 million language. Ultimately, the probability that a monkey \textit{does not} type our specific Shakespeare sequence is indeed $0$, as the amount of "worlds" where the monkey types at random, but does not hit our $A_r(i_1, \cdots, i_k)$ digit sequence is 0.
	
\end{enumerate}

\subsubsection{1.5 Cantor Set Is Trifling, Uncountable, and Perfect}
\begin{enumerate}
	\item The Cantor set $C$ can be defined as the closure of $A_3(1)$. Show that $C$ is uncountable but trifling. First, note uncountable. This can be done by the diagonalization argument. Take any list of numbers in $A_3(1)$, which are also in $C$. We can make a new number in $A_3(1)$, but not in the list, by taking the $ith$ entry, and switching the $0$ for $2$ or $2$ for $0$. Thus, $A_3(1)$ is clearly uncountable, and so to must be $C$.
	\\\\
	Now, we note that $A_3(1)$ is trifling. Take any finite covering of $A_3(1)$ by intervals with total length less than $\epsilon$, than covers $A_3(1)$. Note that we can extend each interval by some length of $\epsilon/2^{i + 3}$ on each edge, and this should cover all of $C$ as well. This is because any element within the closure of $A_3(1)$, can also be viewed as within the closure of all the intervals, and so we can extend the interval lengths a bit. Note, this would apply for every trifling set (in $\R$, not sure about $\R^n$ with covering half open rectangles, but I think the idea could be extended).
	
	\item From $[0,1]$, remove the open middle third $(1/3,2/3)$. From the remainder, a union of two closed intervals, removed the open middle thirds $(1/9,2/9)$ and $(7/9,8/9)$. Show that $C$ is what remains when this process continues ad infimum.
	\\\\
	Note, this is the standard definition of $C$. We have to show this equals our closure definition above. I make the note - the $nth$ step is the closure of points in $[0,1]$ such that the base 3 representation does not contain $1$ in the first $n$ digits. This, I think, is clear. And so, taking the process to infinity, we can conclude that $C$ is the closure of the set where none of the digits are $1$.
	
	\item Show that $C$ is perfect. A set is \textit{perfect} if it is closed and for each $x$ in $A$ and positive $\epsilon$, there is a $y$ in $A$ such that $0 < |x - y| < \epsilon$.
	\\\\
	Note, $C$ is closed, as it is the closure of a set. Now, take $x \in C$ and $\epsilon > 0$. We can find the corresponding $y$ just by matching say the first $n$ digits of $x$, and then flipping the $n + 1$ digit from $0$ to $2$ or vice versa, and then taking any random $0$ or $2$ for the remaining digit. Note, $0 < |x - y| < \epsilon$ if $n$ is large enough. Note - I guess this doesn't apply for the points in the closure. For a point $x \in C$ in the closure, we can find a $y$ in $A_3(1)$ within $\epsilon/2$ of $x$, by the limit definition of the closure. Now, take $z$ within $\epsilon/2$ of $y$ by changing a digit. We now know that $z \neq x$, and the property applies. qed.
\end{enumerate}

\subsubsection{1.6 Alternate $S_n$ Integral Value Proof} We first show the derivative property. We have that:
$$
	M(t) = \int_0^1 e^{ts_n(\omega)} d\omega
$$
We have that $f(\omega,t) = e^{ts_n(\omega)}$. Note, we can make use of \textit{Leibnitz's rule} to take the derivative of $M(t)$ under the integral. See problem 3-32 in Calculus on Manifolds by Spivak. However, this presupposes that $f$ is continuous. I get around this by the following: note that $s_n(\omega)$ has only finite points of discontinuity, when we switch from one dyadic interval of rank $n$ to the next. We can split $\int_0^1$ so that we ignore those points of discontinuity, and only integrate where $s_n(\omega)$ is continuous (and constant). Note, the points of discontinuity have content $0$, and so the integral on $[0,1]$ is equal to the integral on the set not including those points of discontinuity. Thus, we have:
$$
	M'(t) = \int_0^1 D_2(e^{ts_n(\omega)}(0) d\omega =
	\int_0^1 s_n(\omega) e^{t s_n(\omega)} d\omega \implies
	M'(0) = \int_0^1 s_n(\omega) d\omega
$$
Now, we can repeat this operation a finite amount of times, successive differentiation under the integral, to clearly find that:
$$
	M^{(k)}(0) = \int_0^1 s_n(\omega)^k d\omega
$$
Now, noting again that $s_n(\omega)$ is constant on the $2^n$ dyadic intervals, we have that $M(t)$ is actually easy to evaluate. For each of those $2^n$ intervals, $s_n$ is some sum of the form $\pm 1 \pm 1 \cdots \pm 1$, and so we have that:
$$
	M(t) = \frac{1}{2^n} \sum_{i=1}^{2^n} \exp\left(t \left(\pm 1 \pm 1 \cdots \pm 1\right)\right)
$$
We note that this can be broken down with a binomial coefficient. If we have that $k$ of the $\pm 1$ are $-1$, then the value of the sum is $n - 2k$. And so, based off of how many possible sequences of the $\pm 1$ that contain $k$ $-1$, we have that the above can be expressed as:
$$
	= \frac{1}{2^n} \sum_{k=0}^n {n \choose k} \exp(t(n - 2k)) 
	= \frac{1}{2^n} \sum_{k=0}^n {n \choose k} \exp(t(n - k - k))
$$
$$
	= \frac{1}{2^n} \sum_{k=0}^n {n \choose k} \exp(t(n-k) - t(k))
	= \frac{1}{2^n} \sum_{k=0}^n {n \choose k} \exp(t)^{n-k}\exp(-t)^{k}
$$
By the Binomial Theorem, the above equals:
$$
	= \frac{1}{2^n}\left(\exp(t) + \exp(-t)\right)^n = \left(\frac{e^t + e^{-t}}{2}\right)^n =
	\left(\cosh t\right)^n
$$
Where the last step is just an identity. Now, for a new proof of 1.16:
$$
	\int_0^1 s_n(\omega) d\omega = M'(0) = n \cosh^{n-1}(0)\sinh(0) = n * 1 * 0 = 0
$$
Now, for a new proof of 1.18:
$$
	\int_0^1 s_n^2(\omega) d\omega = M''(0) = n \left[(n-1)\cosh^{n-2}(0)\sinh^2(0) + \cosh^n(0)\right] = n * 1 = n
$$
Finally, for a new proof of 1.28, we just have to take the fourth derivative, and plug in 0. I will not be going over the steps, but using derivative calculator, we get the fourth derivative at $0$ is:
$$
	\int_0^1 s_n^4(\omega) d\omega = M''''(0) = n \left(3n - 2\right) = 3n^2 - 2n
$$
Which is the final property. qed.

\subsubsection{1.7 Vieta's Formula} We first find a similar property to the above. We examine:
$$
	\int_0^1 \exp\left[i\sum_{k=1}^n a_kr_k(\omega)\right] d\omega
$$
We note that the summation is constant on the $2^n$ dyadic intervals of rank $n$. In which case, the integral becomes a summation, that looks like:
$$
	= \frac{1}{2^n} \sum \exp\left[i\left(\pm a_1 \pm a_2 \pm \cdots \pm a_n\right)\right]
$$
Now, we extract the first $+a_1$ term and $-a_1$ term from the exponential:
$$
	= \frac{1}{2^n} \exp(ia_1) \sum \exp\left[i\left(\pm a_2 \pm \cdots \pm a_n\right)\right] +
	\frac{1}{2^n} \exp(-ia_1) \sum \exp\left[i\left(\pm a_2 \pm \cdots \pm a_n\right)\right]
$$
We note that by symmetry, both the summations are equal, and so the above actually equals:
$$
	= \frac{1}{2^n} \left(\exp(ia_1) + \exp(-ia_1)\right) \sum \exp\left[i\left(\pm a_2 \pm \cdots \pm a_n\right)\right]
$$
We can continue this process $n$ times, and then split the $2^{-n}$, to find:
$$
	\int_0^1 \exp\left[i\sum_{k=1}^n a_kr_k(\omega)\right] d\omega =
	\prod_{k=1}^n \frac{e^{ia_k}+e^{-ia_k}}{2}
$$
Using the $\cos$ identity, we find:
$$
	\int_0^1 \exp\left[i\sum_{k=1}^n a_kr_k(\omega)\right] d\omega =
	\prod_{k=1}^n \cos(a_k)
$$
We let $a_k = t2^{-k}$. We note that $\sum_{k=1}^\infty r_k(\omega)2^{-k} = 2\omega - 1$ - this is because in each entry, we have $r_k(\omega) = 2d_k(\omega) - 1$, and $\omega = \sum_{k=1}^\infty d_k(\omega) 2^{-k}$. We can apply the $2\omega - 1$ operations (as it is continuous and passes the summation limit), to derive the above. We thus have:
$$
	\lim_{n \to \infty} \int_0^1 \exp\left[i\sum_{k=1}^n a_kr_k(\omega)\right] d\omega =
	\int_0^1 \exp\left[i \lim_{n \to \infty}\sum_{k=1}^n a_kr_k(\omega)\right] d\omega =
	\int_0^1 \exp\left[ti (2\omega - 1)\right] d\omega
$$
Note, we don't have theorems to pass the limit through the integral yet. However, I believe this will be by the monotone convergence theorem - the partial sums should be non decreasing, given the exponential being nonnegative. Anyway, we don't have to prove that here. We now note this is an integral from $0$ to $1$ - whose value is:
$$
	- \frac{ie^{it(2x - 1)}}{2t} = \frac{-i^2}{t}\left(\frac{e^{it} - e^{-it}}{2i}\right) = \frac{\sin(t)}{t}
$$
Taking the limit on the other side, we can conclude:
$$
	\frac{\sin(t)}{t} = \prod_{k=1}^\infty \cos \frac{t}{2^k}
$$
We can now derive Vieta's Formula:
$$
	\frac{2}{\pi} = \frac{\sqrt{2}}{2} \frac{\sqrt{2 + \sqrt{2}}}{2} \frac{\sqrt{2 + \sqrt{2 + \sqrt{2}}}}{2} \cdots 
$$
We recall the half angle formula for $\cos$:
$$
	\cos \frac{\theta}{2} = \sqrt{\frac{1 + \cos(\theta)}{2}}
$$
We can use this to make an induction argument that:
$$
	\cos\left(\frac{\pi}{2} \frac{1}{2^k}\right) = \frac{\sqrt{2 + \sqrt{2 + \cdots + \sqrt{2}}}}{2}
$$
For the base case of $k = 1$, we have:
$$
	\cos\left(\frac{\pi}{2} \frac{1}{2}\right) = \sqrt{\frac{1 + \cos(\pi/2)}{2}} = \sqrt{\frac{1}{2}} = \frac{\sqrt{2}}{2}
$$
Now, for arbitrary $k$, we have:
$$
	\cos\left(\frac{\pi}{2} \frac{1}{2^k}\right) =
	\cos\left(\frac{1}{2}\frac{\pi}{2} \frac{1}{2^{-1}}\right) =
	\sqrt{\frac{1 + \cos\left(\frac{\pi}{2} \frac{1}{2^{k-1}}\right)}{2}} =
	\sqrt{\frac{1 + \frac{\sqrt{2 + \sqrt{2 + \cdots + \sqrt{2}}}}{2}}{2}}
$$
$$
	= \sqrt{\frac{2 + \sqrt{2 + \sqrt{2 + \cdots + \sqrt{2}}}}{4}}
	= \frac{\sqrt{2 + \sqrt{2 + \cdots + \sqrt{2}}}}{2}
$$
And so, by our identity, we find:
$$
	\frac{2}{\pi} = \frac{\sin(\pi/2)}{\pi/2} = \prod_{k=1}^\infty \cos\left(\frac{\pi}{2} \frac{1}{2^k}\right) =
	\frac{\sqrt{2}}{2} \frac{\sqrt{2 + \sqrt{2}}}{2} \frac{\sqrt{2 + \sqrt{2 + \sqrt{2}}}}{2} \cdots 
$$
And we have found Vieta's formula. qed.

\subsubsection{1.8 Differences between the Weak and Strong Law Is Uniform Convergence} A number $\omega$ is normal in base 2 if and only if for each positive $\epsilon$ there exists an $n_0(\epsilon,\omega)$ such that $|n^{-1}\sum_{i=1}^n d_i(\omega) - 1/2| < \epsilon$ for all $n$ exceeding $n_0(\epsilon,\omega)$. That is just the definition. Theorem 1.2 concerns the entire dyadic expansion (ie, the complement of the set of normal numbers whose infinite sum equals 1/2 has probability $0$), whereas Theorem 1.1 concerns only the beginning sequence (ie, the limit of probability, of the dyadic expansion numbers whose first $n$ partial sum is not close to $1/2$, is 0). Identify the difference by showing that for $\epsilon < 1/2$ the $n_0(\epsilon,\omega)$ above cannot be the same for all $\omega$ in $N$ - in other words, $n^{-1}\sum_{i=1}^n d_i(\omega)$ converges to $1/2$ for all $\omega$ in $N$, but not uniformly. But see problem 13.9.
\\\\
Well - noting that for $\epsilon < 1/2$, the $n_0(\epsilon,\omega)$ cannot be the same for all $\omega$ in $N$, means finding $\omega$ in $N$ for which the $n$ is different. We can take just $\omega$ where the first $k$ are all $0$, and then the remaining digits alternate between $0$ and $1$. Note, that the first $n$ for which that sum is actually close to $1/2$ (when normalized by $n^{-1}$), must increase to $\infty$. And so yes, we have convergence to $1/2$ for $\omega$ in $N$, but not uniform convergence. And so, while Theorem 1.1 can rely on this non uniform convergence (ie, the sets are getting smaller, because ultimately, we reach that $n$ value for all $\omega$), we cannot rely on that for Theorem 1.2.
\\\\
Looking at Problem 13.9 - it looks like, however, we might be able to get uniform convergence on subsets of $N$. Perhaps this is how we can actually prove Theorem 1.2. Anyway, I like to just note that the underlying difference in the two Theorems comes from an underlying difference in just real number theory - uniform vs non uniform convergence.

\subsubsection{1.9} 
\begin{enumerate}
	\item Show that a trifling set is nowhere dense. A set $E$ is nowhere dense in $B$ if each open interval $I$ contains some interval $J$ that does not meet $E$. Use the previous problems (1.3 b) and theorems (1.3ii) to prove this.
	\\\\
	I am going to assume we have a trifling set $E$ in the real line. We take an open interval of the real line $I$. We want to find some interval $J$ which our trifling set $E$ does not meet.
\end{enumerate}

\end{document}



